\documentclass[a4paper,12pt]{article}

\usepackage[utf8]{inputenc}
\usepackage[T1]{fontenc}
\usepackage[italian]{babel}
\usepackage{lmodern}
\usepackage{geometry}
\geometry{margin=2.5cm}
\usepackage{hyperref}
\usepackage{float}
\usepackage{graphicx}
\usepackage{amsmath}
\usepackage{amssymb}
\usepackage{array}
\usepackage{booktabs}
\usepackage{xcolor}
\usepackage{listings}
\usepackage{caption}
\usepackage{fancyhdr}
\pagestyle{fancy}
\fancyhf{}
\rhead{DomotiX - Documentazione Tecnica}
\lhead{Team Sviluppo}
\rfoot{\thepage}

\lstset{
  language=JavaScript,
  backgroundcolor=\color{gray!10},
  basicstyle=\ttfamily\footnotesize,
  keywordstyle=\color{blue},
  commentstyle=\color{gray},
  stringstyle=\color{orange},
  breaklines=true,
  showstringspaces=false,
  frame=single
}

\begin{document}
\begin{titlepage}
    \begin{minipage}[t]{0.4\textwidth}
    \raggedright
    {\large\itshape Team di Sviluppo:\par}
    \vspace{0.2cm}
    
    Cociug Raul Andrei - Sviluppatore \\
    Madiotto Gabriel - Sviluppatore
    \end{minipage}
    \hfill
    \begin{minipage}[t]{0.4\textwidth}
    \raggedleft
    {\Large Azienda: DomotiX\par}
    \end{minipage}
    
    \centering
    \vspace*{5cm}
    
    {\Huge\bfseries Documentazione Tecnica\par}
    
    \vfill
    
    \begin{table}[h]
    \centering
    \begin{tabular}{@{}lllc@{}}
    \toprule
    Versione & Data & Autore & Clienti \\  
    \midrule
    1.0 & 12/05/2025 & Cociug, Madiotto & Tollot, Rossi \\
    \bottomrule
    \end{tabular}
    \end{table}
    
    \thispagestyle{empty}
\end{titlepage}

\tableofcontents
\newpage

\section{Introduzione}

Il seguente documento fornisce una descrizione tecnica dettagliata dell'interfaccia grafica della casa domotica. L’interfaccia consente agli utenti di interagire con gli elementi di un’abitazione.

\section{Premessa}

È responsabilità dell’utente disporre di un ambiente di rete compatibile e funzionante con il prodotto, con strumenti consigliati dall'azienda. Le eventuali manutenzioni o modifiche da apportare sono al di fuori dei servizi offerti dall'azienda. Una volta consegnato il prodotto finito al cliente l'azienda non avrà più alcuna responsabilità sul progetto.

\section{Obiettivi del Progetto}

\begin{itemize}
  \item Disporre graficamente un ambiente domotico tramite interfaccia grafica web;
  \item Permettere l'interazione dell'utente con i bottoni e la mappa della casa;
  \item Realizzare un’interfaccia responsive, compatibile con diversi dispositivi;
  \item Gestire lo stato dei dispositivi mediante codice JavaScript locale.
\end{itemize}

\section{Struttura del Progetto}

\begin{itemize}
  \item \texttt{index.html} – struttura principale della pagina;
  \item \texttt{style.css} – gestione grafica e layout delle pagine;
  \item \texttt{buttons.css} – gestione grafica e layout dei bottoni;
  \item \texttt{img-casa.css} – gestione grafica e layout della mappa della casa;
  \item \texttt{script.js} – logica, interazioni e simulazioni dinamiche;
  \item \texttt{/img/} – directory contenente le immagini (icone, mappa, sfondi);
  \item \texttt{/Stanze/} – directory contenente i file HTML delle stanze della casa.
\end{itemize}

\section{Architettura Tecnica}

\subsection{Tecnologie Utilizzate}

\begin{itemize}
  \item \textbf{HTML5}: per la struttura semantica e visiva dell’interfaccia;
  \item \textbf{CSS3}: per la formattazione, animazioni e supporto responsive;
  \item \textbf{JavaScript Vanilla}: per la logica client-side e l’interazione DOM;
  \item \textbf{Cookie API}: per la simulazione di stato persistente lato client.
\end{itemize}

\subsection{Componenti Principali}

\begin{itemize}
  \item \textbf{Mappa interattiva}: ogni stanza o area è cliccabile e mostra controlli dedicati;
  \item \textbf{Moduli funzionali}: luci, porte, meteo, temperatura, Roomba, irrigazione, ora, velocità vento, wattaggio della casa, condizionatore, termosifoni, serrande;
  \item \textbf{Pannello superiore}: titolo della stanza corrente, modalità giorno/notte automatica o forzata;
  \item \textbf{Script dinamico}: ogni modulo ha funzioni dedicate con listener su eventi.
\end{itemize}

\section{Gestione Stato e Interazione}

Ogni elemento cliccabile (es. bottoni luce, porta) è associato a una funzione JavaScript che aggiorna dinamicamente lo stato, applica modifiche al DOM e, dove previsto, salva la preferenza tramite cookie.

\subsection{Esempio - Gestione Luci}

\begin{lstlisting}
// Spegnimento luci
for (let i=0; i < luci.length; i++) {
    luci[i].src = "img/lamp-spenta.png"
}
deleteCookie("btn-luci");

// Accensione luci
for (let i=0; i < luci.length; i++) {
    luci[i].src = "img/lampadina.png";
}
setCookie("btn-luci", "true", 1);
\end{lstlisting}

\subsection{Modalità Giorno/Notte}

L’interfaccia cambia automaticamente colore in base all’orario (06:00-18:00 giorno, altrimenti notte). L’utente può forzare manualmente la modalità tramite un pulsante.

\subsection{Gestione Cookie}

\begin{lstlisting}
// Settaggio di un Cookie
function setCookie(name, value, days) {
    const date = new Date();
    date.setTime(date.getTime() + days * 1000 * 60 * 60 * 24);
    let expires = "expires=" + date.toUTCString();
    document.cookie = `${name}=${value}; ${expires}; path=../`;
}
\end{lstlisting}
Il parametro della funzione "setCookie" name identificherà il nome del Cookie che si vuole creare, value il valore associato a quel nome e days l'ammontare di giorni che si desidera mantenere attivo il Cookie.$\\$
Formato di un Cookie in visualizzazione: "name=value".$\\$
Formato di un Cookie in creazione: "name=value; expires=<UTC Date>; path=./".
\vfill
\begin{lstlisting}
// Prendere valore di un Cookie
function getCookie(name) {
    const cookieArray = document.cookie.split("; ");

    let result = null;
    cookieArray.forEach(element => {
        if(element.indexOf(name) == 0) {
            result = element.substring(name.length + 1);
        }
    });
    return result;
}
\end{lstlisting}
Cicla tutto l'array dei Cookie, precedentemente inizializzato, e se trova un elemento dove il 'name' passato come paramentro si trova all'inizio di esso (indexOf(name) = 0) allora ritorna il valore di quel Cookie.$\\$
\begin{lstlisting}
// Eliminazione Cookie
function deleteCookie(name) {
    setCookie(name, null, null)
}
\end{lstlisting}

\section{Funzionalità Implementate}

\begin{itemize}
  \item Accensione/spegnimento luci (per stanza o globalmente);
  \item Apertura/chiusura porte;
  \item Controllo Roomba (ON/OFF);
  \item Attivazione irrigazione;
  \item Visualizzazione meteo (con temperatura, vento);
  \item Visualizzazione orario attuale;
  \item Modalità responsive adattiva.
\end{itemize}

\section{Responsive Design}

L’interfaccia è stata progettata per adattarsi automaticamente a smartphone, tablet e desktop. L’utilizzo del tag \texttt{meta viewport} e delle \texttt{media queries} CSS permette la fluidità dei contenuti.

\section{Organizzazione del Codice}

\subsection*{HTML}

Layout semplice con tre parti principali (header, mappa della casa, pannello) basato su griglia CSS, con elementi identificabili via ID e classi. Gli elementi principali sono contenuti in div con attributi semantici.

\subsection*{CSS}

Lo stile principale degli elementi HTML si trova nel file style.css, per lo stile dei bottoni si ha buttons.css e infine per la mappa della casa si ha img-casa.css. Per lo stile dinamico e semplificazione di codice si hanno variabili che contengono colori per i vari elementi, questo permette di avere uno stile dinamico con un tema chiaro e un tema scuro.

\subsection*{JavaScript}

Lo script è suddiviso in blocchi per ogni modulo (luci, porte, meteo, ecc.). Ogni blocco contiene:

\begin{itemize}
  \item Funzione \texttt{toggle} o \texttt{update};
  \item Event Listener al caricamento;
  \item Salvataggio dello stato con \texttt{cookie}.
\end{itemize}

\section{Note per Sviluppatori}

\begin{itemize}
  \item Utilizzati nomi coerenti e semantici per ID e classi;
  \item Uso di componenti facilmente scalabili e intuitivi;
  \item Testato su dispositivi mobili e desktop;
  \item Testato su diverse piattaforme e/o browser;
  \item Uso di commenti intuitivi ed esplicativi.
\end{itemize}

\section{Debug e Manutenzione}

\begin{tabular}{|p{5cm}|p{5cm}|p{5cm}|}
\hline
\textbf{Problema} & \textbf{Causa} & \textbf{Soluzione} \\
\hline
Evento non funzionante & ID errato o listener mancante & Controllare ID elemento e listener JS \\
\hline
Luci non si accendono & Immagini assenti o JS corrotto & Verificare classi nel DOM e/o eventuali errori nel JS \\
\hline
Stato non salvato & Cookie disattivati o script errato & Verificare cookie nel browser \\
\hline
UI non responsive & Media query non applicata correttamente & Controllare larghezza dispositivo e/o breakpoints \\
\hline
\end{tabular}

\vspace{0.8cm}
\textit{Per ulteriori chiarimenti, fare riferimento ai commenti all'interno dei file sorgenti.}

\section{Conclusioni}

L'interfaccia della casa domotica sviluppata dalla DomotiX rappresenta un ambiente modulare e ben strutturato, pronto ad essere esteso e integrato. Questa documentazione tecnica è concepita per agevolare futuri sviluppatori nella comprensione, manutenzione e miglioramento del sistema.
\end{document}
