\documentclass[a4paper,12pt]{article}
\usepackage[utf8]{inputenc}
\usepackage[T1]{fontenc}
\usepackage[italian]{babel}
\usepackage{graphicx}
\usepackage{geometry}
\usepackage{fancyhdr}
\usepackage{booktabs}
\usepackage{multicol}
\usepackage[colorlinks=true,
            linkcolor=black,
            citecolor=black,
            urlcolor=blue]{hyperref}

\geometry{a4paper, margin=2.5cm}
\pagestyle{fancy}
\fancyhf{}
\fancyhead[L]{DomotiX}
\renewcommand{\headrulewidth}{0.5pt}
\fancyfoot[C]{\thepage}

\title{Casa Domotica}
\author{Team di Sviluppo}
\date{}

\begin{document}

\begin{titlepage}
    \begin{minipage}[t]{0.4\textwidth}
    \raggedright
    {\large\itshape Team di Sviluppo:\par}
    \vspace{0.2cm}
    
    Cociug Raul Andrei - Sviluppatore 
    Madiotto Gabriel - Sviluppatore
    \end{minipage}
    \hfill
    \begin{minipage}[t]{0.4\textwidth}
    \raggedleft
    {\Large Azienda: DomotiX\par}
    \end{minipage}
    
    \centering
    \vspace*{5cm}
    
    {\Huge\bfseries Casa Domotica\par}
    \vspace{1.5cm}
    
    \vfill
    
    \begin{table}[h]
    \centering
    \begin{tabular}{@{}lllc@{}}
    \toprule
    Versione & Data & Autore & Docenti \\
    \midrule
    1.0 & 31/03/2025 & DomotiX & Tollot, Rossi \\
    \bottomrule
    \end{tabular}
    \end{table}
    
    \thispagestyle{empty}
\end{titlepage}

\tableofcontents
\thispagestyle{fancy}
\newpage

\section{Introduzione}
\subsection{Descrizione Generale del Progetto}
L'obiettivo del progetto è quello di sviluppare un sistema che permetta il controllo e il monitoraggio di vari dispositivi domestici attraverso un'interfaccia utente interattiva.
Il progetto si compone di due principali elementi:\begin{itemize}
    \item \textbf{La rete della casa domotica}, realizzata mediante il software Cisco Packet Tracer, per simulare la comunicazione tra i vari dispositivi connessi.
    \item \textbf{L'interfaccia utente}, sviluppata con HTML, CSS e JavaScript, per consentire l'interazione con i dispositivi attraverso un mockup funzionale.\end{itemize}

\subsection{Informazioni Progetto}
\begin{itemize}
    \item \textbf{Data Inizio Progetto:} 11 Marzo 2025
    \item \textbf{Data Fine Prevista:} 14 Maggio 2025
    \item \textbf{Durata Stimata:} 65 giorni/ 9 settimane
    \item \textbf{Partecipanti:}
    \begin{itemize}
        \item Cociug Raul
        \item Madiotto Gabriel
    \end{itemize}
\end{itemize}

\section{Situazione Attuale}
\subsection{Stato Corrente}
È stato sviluppato il mockup con HTML, CSS, JavaScript per mostrare l'anteprima dell'interfaccia utente che dovrà essere utilizzata dagli utenti per poter interagire con la casa domotica e i dispositivi connessi ad essa.
È stato realizzato un pannello di controllo, per la casa e tutte le stanze, con il quale è possibile svolgere numerose funzioni come:\begin{itemize}
\item Accendere/spegnere le luci
\item Controllo della temperatura
\item Controllo della velocità del vento
\item Verifica dello stato delle porte (aperte/chiuse)
\end{itemize}
È stato realizzato anche il questionario da dover porre ai clienti per l'analisi dei requisiti e delle funzionalità da implementare per personalizzare il prodotto come richiesto dal cliente.

\subsection{Materiale Esistente}
\begin{itemize}
\item \textbf{Mockup/Interfaccia utente}
\item \textbf{Questionario}
\item \textbf{Documentazione del progetto}
\end{itemize}

\section{Requisiti Hardware e Software}
\subsection{Requisiti del Cliente}
\begin{itemize}
    \item \textbf{Requisiti Hardware:}
    \begin{itemize}
        \item Dispositivi per l'utilizzo dell'interfaccia come smartphone, tablet o computer (fisso o portatile)
    \end{itemize}
    
    \item \textbf{Requisiti Software:}
    \begin{itemize}
        \item Applicativi per l'utilizzo dell'interfaccia/IDE (es. VS Code)
        \item Cisco Packet Tracer per utilizzo della rete
    \end{itemize}
\end{itemize}

\section{Requisiti Funzionali}
\subsection{Richieste Cliente HMI}
Il cliente ha richiesto un'interfaccia semplice, intuitiva e facilmente utilizzabile per controllare i dispositivi della casa domotica e interagire con il sistema realizzato.

\subsection{Specifiche Funzionali}
Il sistema deve soddisfare requisiti come:\begin{itemize}
    \item Controllo meteo, velocità del vento e tempo
    \item Rilevazione energia totale utilizzata e temperatura all'interno della casa domotica
    \item Gestione porte con lucchetto da remoto (aprire/chiudere)
    \item Controllo e gestione delle lampadine (accendere/spegnere)
    \item Irrigazione delle piante
    \item Controllo e gestione delle serrande (alzare/abbassare)
    \item Accensione e spegnimento delle serrande
    \item Accensione e spegnimento dell'aria condizionata
    \item Gestione dell'aspirapolvere automatico (Roomba)
\end{itemize}

\subsection{Progettazione Interfaccia Grafica (GUI)}
\begin{itemize}
    \item \textbf{Il Layout}, formato da un header con l'opzione di modalità notturna, la casa domotica sulla destra e il pannello di controllo con tutte le funzionalità previste per il sistema.
    La disposizione degli elementi è stata ottimizzata per schermi di diverse dimensioni, implementando un design responsive che si adatta automaticamente a tablet e dispositivi mobili.
    \item \textbf{Elementi Grafici}, i quali sono stati selezionati e progettati per garantire coerenza visiva e migliorare la comunicazione con l'utente:\begin{itemize}
        \item I colori, utilizzando un tema per il giorno, con tonalità azzurre su sfondo bianco, e un tema per la notte, con colori come grigio e nero.
        \item Tipografia, sono stati utilizzati caratteri Arial per migliorare la leggibilità su tutti i dispositivi.
        \item Iconografia, sono state utilizzate icone semplici da comprendere per gli utenti.
        \item Componenti interattivi come pulsanti, campi di input e controlli sono stati progettati con stati visivi chiari (normale, hover, attivo, disabilitato) per comunicare efficacemente il loro stato all'utente.
    \end{itemize}
\end{itemize}

\subsection{Progettazione Rete}
Il sistema domotico richiede una rete per la comunicazione tra i dispositivi. La progettazione della rete include:\begin{itemize}
    \item Corretto posizionamento dei dispositivi di rete
    \item Controllo e monitoraggio della rete per la rilevazione di eventuali malfunzionamenti
\end{itemize}

\end{document}